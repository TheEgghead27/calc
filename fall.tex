\documentclass{article}
\usepackage{graphicx} % Required for inserting images
\usepackage{amsfonts} % for mathbb
\usepackage{amsmath}
\usepackage{hyperref}

\title{Honner's Honors Calculus (Fall Semester)}
\author{David Chen}
\date{September 2023 - January 2024}

\begin{document}

\maketitle

\section{Functions}
Functions map a domain (a set of $x$ ``inputs'') to one and only one element of the range a set of ($y$ ``outputs''). Usually, we are discussing functions of \underline{x} to \underline{y}.
Functions are a subset of Relations (which relate a set of inputs to at least 1 output -- see \ref{vertical-line}).

\subsection{Representations of a Function}
There are many ways to represent a function, such as a
\begin{itemize}
    \item Table
    \item Graph
    \item Rule (including piecewise representations)
    \item Set
\end{itemize}

\subsection{Finding the Range/Domain}
For our purposes, we will usually be dealing only with real-valued functions.

A common domain will be $x \in \mathbb{R}$, all real numbers.

In this case, for every vertical line ($x=c$), that line will intercept the graph (there is a value for the function defined at that point).

Similarly, to show that the range is $y \in \mathbb{R}$, use horizontal lines ($y=c$). In general, the range will be dependent on the domain.

"Holes" in the domain, represented by open circles in graphs, are artifacts of the exclusion of some value from the domain.

\subsection{The Vertical Line Test} \label{vertical-line}
To verify that a graph of a relation is a function, one can use the vertical line test.

If for any vertical line ($x=c$), there are multiple intersections with the relation, the relation is NOT a function. Otherwise, the relation is a function.

\subsubsection{The Horizontal Line Test}
To verify that a function is invertible, we check if any horizontal line ($y=c$) has multiple intercepts with the graph of the function. If not, every output ($c$) will correspond with one input $x$, so $f^{-1}(y)=x$ is a valid function.

\subsection{Transformations}
Transformations alter a function. They can be composed, but function composition is not commutative,\footnote{Subtraction is anti-commutative, since switching the order of the terms will make the result negative.} so the order matters.


\subsubsection{f(x + c)}
This shifts (translates) the domain by $c$ ("moving left $c$" on the graph).
\subsubsection{f(x) + c}
This shifts (translates) the range by $c$ ("moving up $c$" on the graph).
\subsubsection{f(-x)}
This alters the domain ("reflecting over the y-axis" on the graph).
\subsubsection{-f(x)}
This alters the range ("reflecting over the x-axis" on the graph).
\subsubsection{cf(x)}
This is an axial scaling, for only one dimension is changed.

This is not a translation, nor is it a dilation (which is based on a center point and a scale factor).

\section{Limits}
If the difference between two numbers is smaller than can be imagined (arbitrarily/indistinguishable close), then the difference is $0$, and they are the same number.
$$\lim_{x\to a}f(x)=L$$
As $x$ approaches $a$ (in the domain), $f(x)$ gets arbitrarily close to $L$ (in the range).

We say that $\lim_{x\to a}\limits f(x)$ exists if
\begin{enumerate}
    \item $\lim_{x\to a^+}\limits f(x)$ exists (approaching from the positive side)
    \item $\lim_{x\to a^-}\limits f(x)$ exists (approaching from the negative side)
    \item $\lim_{x\to a^+}\limits f(x) = \lim_{x\to a^-}\limits f(x)$
\end{enumerate}

Limits may not exist for odd reasons, such as the domain being integers ($x\in\mathbb{Z}$), since limits are meant to be connected paths through the domain.

In addition, the function may exist as points the limit of the function does not, while the limit may exist at points the function may not.

\subsection{The (Delta-Epsilon) Definition of a Limit}
For a function $f(x)$ and a real number L, the statement $\lim_{x\to a}\limits f(x) = L$ means that for every $\epsilon > 0$, there exists a $\delta > 0$ such that $0 < |x-a| < \delta$ implies that $|f(x) - L| < \epsilon$.

$\epsilon$ (epsilon) represents a tolerance in the range, and $\delta$ (delta) represents a tolerance in the domain.

\subsection{Limit Laws}\label{limit-laws}
Limits are not actually numbers, but behave similarly to numbers in that there are algebraic properties related to them.

For instance, letting $b,c \in \mathbb{R}$, $n \in \mathbb{N}$, and
\begin{align*}
    \lim_{x\to c} f(x) = L  & & \lim_{x \to c} g(x) = K
\end{align*}

Then the following properties hold:
\begin{enumerate}
    \item Scalar multiple\quad$\lim_{x\to c}\limits\left[bf(x)\right] = bL$
    \item Sum/Difference\quad$\lim_{x\to c}\limits\left[f(x)\pm g(x)\right] = L \pm K$
    \item Product\quad$\lim_{x\to c}\limits\left[f(x)g(x)\right] = LK$
    \item Quotient\quad$\lim_{x\to c}\limits\left[f(x) / g(x)\right] = L / K$, if $K\ne 0$
    \item Power\quad$\lim_{x\to c}\limits\left[f(x)\right]^n = L^n$
\end{enumerate}

\subsection{Infinity}
Infinity is greater than any real number.

$$\lim_{x\to a}f(x) = \infty$$ means that as $x$ gets arbitrarily close to $a$, the value of $f(x)$ grows without bound. For any $N>0$, $\exists\delta$ s.t. $0 < |x-a| < \delta$ implies that $f(x) > N$.

To say that a limit $= \infty$ is to say that this limit does not exist in a particular way (and so the Limit Laws do not apply).

$$\lim_{x\to\infty}f(x) = L$$ means that for every $\epsilon > 0$ there exists an $N > 0$ such that $x>N$ implies that $|f(x)-L|<\epsilon$.

\subsection{Indeterminate Forms}
The indeterminate forms $\frac{0}{0}$ and $\frac{\infty}{\infty}$ are undefined, they could be anything.

For instance:
\begin{align*}
\lim_{x\to 0} \frac{\sin{x}}{x} = 1 & & \lim_{x\to 0}\frac{x}{x^2} = \infty
\end{align*}

This is not about how small/large the values get, but how fast the values get smaller/larger.

\subsection{Polynomials}
Based on the Limit Laws (\ref{limit-laws}), and the fact that $\forall c \lim_{x\to c}\limits x = c$, every polynomial\footnote{or rational function so long as f(c) exists} has the property that $\forall c \lim_{x\to c}\limits f(x)  = f(x)$.

\subsubsection{Rational Functions}
Rational functions are all of the form $\frac{P(x)}{Q(x)}$, with $P, Q$ being polynomials where
\begin{align*}
    P(x) = a_nx^n + a_{n-1}x^{n-1} + \ldots + a_1x + a_0 \\
    Q(x) = b_nx^n + b_{n-1}x^{n-1} + \ldots + b_1x + b_0
\end{align*}

$$\lim_{x\to\infty}\frac{P(x)}{Q(x)} = \begin{cases}
    0 & deg(q) > deg(p) \\
    \frac{a_n}{b_n} & deg(q) = deg(p)\\
    \pm \infty & deg(q) < deg(p)
\end{cases}$$
The last case depends on the signs of $a_n$ and $b_n$.

When $x$ is infinitely large, all that matters is the leading degree term, but when $x$ is small (near $0$), the smaller degree terms matter more.

\subsection{Asymptotes}
Any function can be an asymptote to another function. The asymptote need not appear at $\pm\infty$, such as in the case of vertical asymptotes (e.g. $y=\log{x}$ has a vertical asymptote $x=0$), and the graph of the function may intersect the asymptote at times.

\subsection{Important Theorem I}
If $f$ and $g$ agree everywhere, except possibly at $x=a$, then \[ \lim_{x \to a} f(x) = \lim_{x \to a} g(x) \] assuming these limits exist.

This allows us to treat rational functions like polynomials.

\subsection{The Squeeze Theorem}
Suppose $h(x) \le f(x) \le g(x)$ $\forall x \in$ an open interval $I$ that contains $x=c$, except possibly $x=c$ itself. If $\lim_{x\to c}\limits h(x) = L = \lim_{x\to c}\limits g(x)$, then $\lim_{x \to c}f(x) = L.$

\subsection{Special Trig Limits}\label{trig-limits}
$$\lim_{x\to 0} \frac{\sin{x}}{x} = 1$$ or in engineer terms, $\sin{x} \approx x$ for small $x$.

This is proven by the fact that $\sin{\theta}\cos{\theta} \le \theta \le \frac{\sin{\theta}}{\cos{\theta}}$. The former statement is based on the area of the arc compared to the area of the right-triangle, and the latter statement is based on the length of the arc compared to the right-triangle with the tangent line.

Because we are working in $0 \le \theta \le \frac{\pi}{2}$, $\sin{\theta} > 0$, we can flip this expression around and show via the Squeeze Theorem that the limit is 1.

$$\lim_{x\to 0} \frac{1-\cos{\theta}}{\theta} = 0$$

\subsection{Continuity}
The function $f(x)$ is continuous at $x=c$\footnote{Notice that continuity must be a local property about a point.} if
\begin{enumerate}
    \item f(c) exists
    \item the limit $\lim_{x\to c}\limits f(x)$ exists
    \item $\lim_{x\to c}\limits f(x) = f(c)$
\end{enumerate}
A function $f(x)$ is continuous on a set $S$ if $\forall a \in S$, $f(x)$ is continuous at $x=a$.\footnote{This definition may be adjusted for endpoints to use only one-sided limits.} To say that ``$f$ is continuous'' is usually to say that $f$ is continuous on $\mathbb{R}$.

Informally, a continuous function can be drawn without lifting one's pencil.

\subsubsection{Discontinuities}
\paragraph{Point Discontinuity} One point is missing in the curve. This is also called a removable discontinuity because adding that one point can fix the discontinuity.
\paragraph{Jump Discontinuity} There is a vertical gap between one portion of a function and the next.
\paragraph{Infinite Discontinuity} The function goes to $\pm \infty$ as $x \to c$.

\subsubsection{Continuous Functions}
\begin{itemize}
    \item All polynomials are continuous.
    \item All rational functions are continuous where they are defined.
    \item All trigonometric functions are continuous where they are defined.
\end{itemize}

If $f,g$ are continuous, then $f+g$, $fg$, $\frac{f}{g}$ ($g\ne0$), and $cf$ are all continuous. The composition of continuous functions is continuous.

If $f(x)$ is continuous at $x=b$ and $\lim_{x\to a}g(x) = b$, then $\lim_{x\to a}f(g(x)) = f(b)$. Limits can pass through continuous functions.

\subsubsection{The Intermediate Value Theorem (IVT)}
Suppose $f(x)$ is continuous on the \underline{closed} interval $[a,b]$, and $f(a)\ne f(b)$. If $f(a) < k < f(b)$, $\exists c \in (a,b)$ s.t. $f(c)=k$. There may be multiple such intersections, but we only require one.

\paragraph{Related Theorem A}
Suppose $f(x)$ is continuous on $[a,b]$. If $f(x)$ has no zeroes on the interval, then $f$ is either entirely positive or entirely negative on the interval.

\paragraph{Related Theorem B}
Suppose $f(x)$ and $g(x)$ are continuous on $[a,b]$. If $f(a) < g(a)$ and $f(b) > g(a)$, then $\exists c$ s.t. $f(c) = g(c)$, based on the IVT for the continuous function $f-g$.

\section{Derivatives}
\subsection{Secant Lines}
A secant line is the line joining two points $(a, f(a))$ and $(b, f(b))$ on the graph of a function. Its slope is $\frac{f(a)-f(b)}{a-b}$.

This slope can be considered the average rate of change of the function between $a$ and $b$, or an approximation to the derivative.


\subsection{Tangent Lines}
If $f(x)$ is defined on an open interval containing $c$ and if $\lim_{x\to c}\limits \frac{f(x) - f(c)}{x-c}=m$ exists, then the line passing through $(a, f(a))$ with slope $m$ is the tangent line to the graph of $y=f(x)$ at $x=c$. In this case, we write $f'(c) = m$ (f prime of c) can call $f'(c)$ the derivative of $f(x)$ at $x=c$.

A tangent line is the best local linear approximation to a function at a point. Thus, given the derivative at a point $(x, f(x))$, one can approximate $f(x\pm a) \approx f'(x)(a) + f(x)$.

Tangency is a property of a graph, not a function.

A linear approximation is usually far easier to work with compared to the original form of a function.

\subsubsection{The Vertical Difference Method}
If $g(x)=mx+b$ is tangent to $f(x)$ at $x=a$, then we expect $V(x) = f(x) - g(x)$ to have a double root at $x=a$.

This places us at the unstable singularity between the region with no intersections and 2 intersections.

Notice that $P(x) = (x-a)^2Q(x) + (mx+b)$, the remainder of dividing a polynomial $P(x)$ by $(x-a)^2$ provides the tangent line at $x=a$.
\subsection{Definition of the Derivative}
Differentiability is a local property. ``$f$ is differentiable at $x=c$'' if the limit (described below) exists.
\paragraph{At a point}
$$f'(c) = \lim_{x\to c} \frac{f(x) - f(c)}{x-c}$$

\paragraph{As a function}
$$f'(x) = \lim_{\Delta x\to 0} \frac{f(x) - f(x+\Delta x)}{\Delta x}$$

\paragraph{Symmetric Difference Quotient}
This is not quite the same as the derivative.
$$\lim_{h\to 0} \frac{f(x+h) - f(x-h)}{2h}$$

\paragraph{Leibniz Notation}
$$\frac{dy}{dx}$$
\subsubsection{Important Theorem II}
Differentiability implies continuity.\\
If $f(x)$ is differentiable at $x=c$, then $f(x)$ is continuous at $x=c$.

$|x|$ is a good example of a continuous function that is not differentiable, due to its cusp.

$x^\frac{1}{3}$ is also not differentiable at $x=0$, due to the fact that the tangent line is vertical.

\paragraph{Right/Left Differentiability}
If we wished to define a function to be differentiable on a closed interval $[a,b]$, we would have to use one-sided limits.


\subsection{Differentiation Rules}
\begin{itemize}
    \item $\frac{d}{dx}\left[ax+b\right] = a$\quad"The derivative of a linear line is its slope."
    \item $\frac{d}{dx}\left[c\right] = 0$\quad"The derivative of a constant function is 0.
    \item $\frac{d}{dx}\left[f(x) \pm g(x)\right] = \frac{d}{dx}\left[f(x)\right] \pm \frac{d}{dx}\left[g(x)\right]$
    \item $\frac{d}{dx}\left[cf(x)\right] = c\frac{d}{dx}\left[f(x)\right]$
    \item $\frac{d}{dx}\left[x^n\right] = nx^{n-1}$\quad The Power Rule -- proven using the Binomial Theorem for integers, but applicable to all real exponents.
    \item $\frac{d}{dx}\left[\sin{x}\right] = \cos{x}$\quad Proven using the limits from \ref{trig-limits}.
    \item $\frac{d}{dx}\left[\cos{x}\right] = -\sin{x}$
    \item $\frac{d}{dx}\left[\tan{x}\right] = \frac{1}{\cos^2{x}} = \sec^2{x}$
    \item $\frac{d}{dx}\left[\cot{x}\right] = \frac{-1}{\sin^2{x}} = -\csc^2{x}$
\end{itemize}

\subsubsection{The Product Rule}
$$\frac{d}{dx}\left[f(x)g(x)\right] = f(x)g'(x) + g(x)f'(x)$$
"First times the derivative of the second plus the second times the derivative of the first."

\subsubsection{The Quotient Rule}
$$\frac{d}{dx}\left[\frac{p}{q}\right] = \frac{d}{dx}\left[p\cdot q^{-1}\right] = \frac{g(x)f'(x) - f(x)g'(x)}{\left(g(x)\right)^2}$$
"Low $d$ high minus high $d$ low over low squared."

\subsubsection{The Chain Rule}
$$\frac{d}{dx}\left[f(g(x))\right] = f'(g(x))g'(x)$$

\subsection{Implicit Differentiation}
If you cannot solve for an explicit form of $y=f(x)$, you can still differentiate to find an expression for $\frac{dy}{dx}$ (in terms of $x$ and $y$) by differentiating the two sides of an equation, considering the chain rule.

For instance $\sin{y}=x$ has derivative $\frac{dy}{dx}=\sec{y}$ from implicit differentiation.

This method is helpful in solving related rates problems.

\subsection{Applications of Differentiation}
\subsubsection{Increasing/Decreasing}
If $f'(x) > 0$, $f(x)$ is increasing. If $f'(x) < 0$, $f(x)$ is decreasing.


\section{Appendix}
\subsection{Notation}
\subsubsection{\texorpdfstring{The Universal Quantifier $\forall$}{The Universal Quantifier}}
This symbol stands for "for all ..."

For instance, $\forall x$ means "for all x", the following statement is true.

To describe that a function is odd, one can write $\forall x \left(f\left(-x\right) = -f\left(x\right)\right)$.

\subsubsection{\texorpdfstring{The Existential Qualifier $\exists$}{The Existential Qualifier}}
This symbol stands for "there exists ..."

For instance $\exists x$ means that there exists (at least one value of) $x$ such that the following statement is true.

\subsubsection{Sets}
$\left\{\right\}$ represent set builder notation.

$\in$ is the "in" symbol.

$|$ represents "such that".

$\cup$ represents a union of sets (inclusive "or").\\
Example: $\left\{x \in \mathbb{R} | 1 \le x \le 5 \right\}$ and $\left\{\theta \in \mathbb{R} | 1 \le \theta \le 5 \right\}$ represent the same set, but with different labels.

$\subset$ and $\subseteq$ are the subset (or equal to) symbols.

\subsection{Additional Functions}
\subsubsection{ceil(x)}
The ceiling function returns the smallest integer ($y\in\mathbb{Z}$) such that $y \ge x$.\\
View a graph on \href{https://www.desmos.com/calculator/cpay9r9g5w}{Desmos}.

\subsubsection{sgn(x)}
The sign function. Defined as
sgn$(x) = \frac{x}{|x|}$ or sgn$(x) = \begin{cases}
    -1 &x<0\\
    +1 &x>0
\end{cases}$\\
A similar function, the pulse function, defines pulse$(0)=0$.

\subsubsection{The Dirichlet Function}
This is a pathological function constructed to be nowhere continuous.

D$(x) = \begin{cases}
    1 & x\in\mathbb{Q}\\
    0 & x\notin\mathbb{Q}
\end{cases}$

\subsection{Circle Facts}
\begin{itemize}
    \item Subtends is a word for the relationship between a central angle and its arc.
    \item All circles are similar.
    \item To find the arclength $s$ of a circle's arc from the central angle measure $\theta$ and the circle's radius $r$: $s=\theta r$.
    \item The sector area of a circle is $A=\frac{1}{2}r^2\theta$.
\end{itemize}

\subsection{Additional Area Formulae}
\begin{itemize}
    \item Triangle with angle C and sides a, b -- $A=\frac{1}{2}ab\sin{C}$
    \item Rhombus with angle C and side length a -- $A=a^2\sin{C}$
\end{itemize}

\end{document}
